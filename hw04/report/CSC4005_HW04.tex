\documentclass[twoside,12pt]{article}
\usepackage[left=1in, right=1in, top=1in, bottom=1in]{geometry}
\usepackage{amsmath}
\usepackage{amssymb}
\usepackage{amsfonts}
\usepackage{mathtools}
\usepackage{amsthm}
\usepackage{fancyhdr}
\usepackage{enumitem}
\usepackage{siunitx}
\usepackage{booktabs}
\usepackage[hidelinks]{hyperref}
\usepackage{sectsty}
\usepackage{mathrsfs} % mathscr
\usepackage{tikz}
\usepackage{pgfplots}
\usepackage{multicol}
\usepackage{listings}
% \usepackage{amsart}
\usepackage{fontspec}
\usepackage{titlesec}
\usepackage{subcaption}

% less hyphens, similar to tolerance?
\usepackage{microtype}

% allow H option of figure
\usepackage{float}

% math font (libertine)
\usepackage{libertinus-otf}

% braket
\usepackage{braket}

% mono font
\usepackage{inconsolata}
\setmonofont{inconsolata}

% define Latin modern font environment
\newcommand{\lms}{\fontfamily{lmss}\selectfont} % Latin Modern Roman
% \newcommand{\lmss}{\fontfamily{lmss}\selectfont} % Latin Modern Sans
% \newcommand{\lmss}{\fontfamily{lmtt}\selectfont} % Latin Modern Mono

% % change mathcal shape
% \usepackage[mathcal]{eucal}


% define math operators
\newcommand{\FF}{\mathbb{F}}
\newcommand{\RR}{\mathbb{R}}
\newcommand{\NN}{\mathbb{N}}
\newcommand{\ZZ}{\mathbb{Z}}
\newcommand{\QQ}{\mathbb{Q}}
\newcommand{\XX}{\mathbb{Y}}
\newcommand{\CL}{\mathcal{L}}
% \renewcommand{\d}{\mathrm{d}}
\renewcommand*\d{\mathop{}\!\mathrm{d}}
\DeclareMathOperator*{\argmax}{arg\,max}
\DeclareMathOperator*{\argmin}{arg\,min}
\DeclareMathOperator{\im}{im}
\DeclareMathOperator{\id}{id}
\renewcommand{\mod}[1]{\ (\mathrm{mod}\ #1)}

% section font style
\sectionfont{\lms\large}
\subsectionfont{\lms\normalsize}
\subsubsectionfont{\bf}

% line spreading and break
\hyphenpenalty=5000
\tolerance=20
\setlength{\parindent}{0em}
\setlength\parskip{0.5em}
\allowdisplaybreaks
\linespread{0.9}

% theorem
% latex theorem
% definition style
\theoremstyle{definition}
\newtheorem{theorem}{Theorem}[subsection]
\newtheorem{axiom}{Axiom}[section]
\newtheorem{definition}{Definition}[section]
\newtheorem{example}{Example}[section]
\newtheorem{question}{Question}[section]
\newtheorem{exercise}{Exercise}[section]
\newtheorem*{exercise*}{Exercise}
\newtheorem{lemma}{Lemma}[section]
\newtheorem{proposition}{Proposition}[section]
\newtheorem{corollary}{Corollary}[section]
\newtheorem*{theorem*}{Theorem}
\newtheorem{problem}{Problem}
% remark style
\theoremstyle{remark}
\newtheorem*{remark}{Remark}
\newtheorem*{solution}{Solution}
\newtheorem*{claim}{Claim}


% paragraph indent
\setlength{\parindent}{0em}
\setlength\parskip{0.5em}

\newcommand\Code{CSC4005 FA22}
\newcommand\Ass{HW04}
\newcommand\name{Haoran Sun}
\newcommand\mail{haoransun@link.cuhk.edu.cn}

\title{{\lms \Code \ \Ass}}
\author{\lms \name \ (\href{mailto:\mail}{\mail})}
\date{\sffamily \today}

\makeatletter
% \let\Title\@title
\let\theauthor\@author
\let\thedate\@date

\fancypagestyle{plain}{%
    \fancyhf{}
    \lhead{\sffamily \Ass}
    \rhead{\sffamily \name}
    \rfoot{\sffamily\thepage}

    % # 页脚自定义
    \fancyfoot[L]{
        \begin{minipage}[c]{0.06\textwidth}
            \includegraphics[height=7.5mm]{logo2.png}
        \end{minipage}
    }
}
\fancypagestyle{title}{%
    \fancyhf{}
    \renewcommand{\headrulewidth}{0pt}
    % \lhead{\Title}
    % \rhead{\theauthor}
    \rfoot{\sffamily\thepage}

    % # 页脚自定义
    \fancyfoot[L]{
        \begin{minipage}[c]{0.06\textwidth}
            \includegraphics[height=7.5mm]{logo2.png}
        \end{minipage}
    }
}
\fancyfootoffset[L]{0.3cm}

% re-define title format
% \makeatletter
% \renewcommand{\maketitle}{\bgroup\setlength{\parindent}{0pt}
% \begin{flushleft}
%   \textbf{\Large\@title}
%   \@author
% \end{flushleft}\egroup
% }
% \makeatother
\makeatletter
\renewcommand{\maketitle}{\bgroup\setlength{\parindent}{0pt}
\begin{center}
  \textbf{\Large\@title}\\
  \@author
\end{center}\egroup
}
\makeatother

\pagestyle{plain}

% lstlisting settings
\lstset{
    basicstyle=\linespread{0.8}\ttfamily\small,
    breaklines=true,
    basewidth=0.5em,
    frame=single,
}
\lstdefinestyle{output}{
    basicstyle=\linespread{0.8}\ttfamily\footnotesize,
    breaklines=true,
    basewidth=0.5em,
    frame=single,
}    
\lstdefinestyle{sh}{
    basicstyle=\linespread{0.8}\ttfamily\footnotesize,
}
\lstdefinestyle{cpp}{
    numbers=left,
    basicstyle=\linespread{0.8}\ttfamily\footnotesize,
    numberstyle=\linespread{0.8}\ttfamily\footnotesize,
    language=C++,
    xleftmargin=6.0ex,
    frame=single,
}


\begin{document}
\begin{titlepage}
    \maketitle
    \thispagestyle{title}
\end{titlepage}

\section{Introduction}




\section{Method}
\subsection{System setup}


\subsection{Program design and implementation}


\subsection{Usage}


\subsection{Performance evaluation}
The program was executed under 
different configurations to evaluate performance.
With 20 different CPU core numbers (from 1 to 20 with increment 1, $p=1, 2,\dots, 20$)
and 20 different $n$ (from 50 to 1000 with increment 50),
400 cases in total were sampled for sequential, MPI, OpenMP, and Pthread programs.
Test for CUDA program is implemented separately since GPU is much faster
than all CPU programs and only large-scale performance will be discussed
on CUDA program.
Recorded runtime is analyzed through the Numpy
package in Python.
Figures were plotted through the Matplotlib and the Seaborn packages in Python.
Analysis codes were written in \lstinline|analysis/main.ipynb|.


\newpage
\section{Result and discussion}


\subsection{CPU parallelization}


\subsection{GPU parallelization}


\section{Conclusion}
In conclusion, four parallel computing schemes for $n$-body simulation 
are implemented and their performances are evaluated.
For large, ignoring the precision, one may use GPU to accelerate
the calculation.


\appendix
% \renewcommand\thefigure{\thesection.\arabic{figure}}
\counterwithin{figure}{section}

\newpage
\section{Supplementary figures}



\clearpage
\newpage
\section{Source code}
% \lstinputlisting[style=cpp,language=,title=\lstinline|CMakeLists.txt|]{../CMakeLists.txt}
% \lstinputlisting[style=cpp,language=,title=\lstinline|src/CMakeLists.txt|]{../src/CMakeLists.txt}
% \lstinputlisting[style=cpp,title=\lstinline|src/main.cpp|]{../src/main.cpp}
% \lstinputlisting[style=cpp,title=\lstinline|src/main.mpi.cpp|]{../src/main.mpi.cpp}
% \lstinputlisting[style=cpp,title=\lstinline|src/cudalib.cu|]{../src/cudalib.cu}
% \lstinputlisting[style=cpp,title=\lstinline|src/utils.h|]{../src/utils.h}
% \lstinputlisting[style=cpp,title=\lstinline|src/utils.cuh|]{../src/utils.cuh}
% \lstinputlisting[style=cpp,title=\lstinline|src/const.h|]{../src/const.h}





% \end{multicols*}
\end{document}

